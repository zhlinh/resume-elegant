% !TEX TS-program = xelatex
% !TEX encoding = UTF-8 Unicode
% !Mode:: "TeX:UTF-8"

\documentclass{resume}
\usepackage{zh_CN-Adobefonts-External} % Simplified Chinese Support using external fonts (./fonts/zh_CN-Adobefonts/)
%\usepackage{zh_CN-Adobefonts-Internal} % Simplified Chinese Support using system fonts
\usepackage{linespacing_fix} % disable extra space before next section
\usepackage{cite} % bibliology

\begin{document}
\pagenumbering{gobble} % suppress displaying page number

\name{黄振林}

% {E-mail}{mobilephone}{homepage}
% be careful of _ in emaill address
\contactInfo{zhlinhng@gmail.com}{(+1) 650-667-5006}{http://www.zhenlin.me}
% {E-mail}{mobilephone}
% keep the last empty braces!
%\contactInfo{xxx@yuanbin.me}{(+86) 131-221-87xxx}{}

\section{\faGraduationCap\ 教育背景}
\datedsubsection{\textbf{西安电子科技大学},陕西}{2014年 -- 至今}
\textit{研究生在读}\ 电子与通信工程,预计 2017年6月毕业
\datedsubsection{\textbf{西安电子科技大学},陕西}{2010年 -- 2014年}
\textit{学士}\ 通信工程

\section{\faTrophy\ 获奖情况}
\begin{onehalfspacing}
\datedline{校学业奖学金一等奖,\textbf{专业均分: 90.29/100,排名: 3/265}}{2015年11月}
\datedline{全国计算机等级考试四级网络工程师}{2012年12月}
\datedline{大学生数模竞赛校一等奖}{2012年05月}
\end{onehalfspacing}

\section{\faCogs 相关技能}
% increase linespacing [parsep=0.5ex]
\begin{itemize}[parsep=0.5ex]
  \item \textbf{编程语言:} \textbf{Java, Android,} (HTML, CSS, JavaScript, MySQL, Python)
  \item \textbf{系统平台:} Linux, OpenWrt嵌入式系统, Windows
  \item \textbf{开发工具:} Vim, AndroidStudio
  \item \textbf{英语水平:} CET-6 (449)
\end{itemize}

\section{\faUsers\ 项目经历}
\datedsubsection{\textbf{PTP-1588虚拟机对时系统}}{2015年9月 -- 至今}
\role{小组组长}{中船七〇九所项目}
\setlength{\parindent}{2em}提高虚拟机时间同步精度
\begin{onehalfspacing}
\begin{itemize}[leftmargin=5em]
  \item 基于IEEE 1588协议,针对虚拟集群的特性与需求,提出相应的时间同步方案,并使该方案满足同步精度要求。
  \item 基于时间服务器和PTP板卡提出并实现同步测量方案。
  \item 在PTPd软件同步的基础上针对虚拟机对同步效果进行优化。
  \item 实现支持单物理机上同时运行多虚拟机的硬件同步方案。
\end{itemize}
\end{onehalfspacing}

\datedsubsection{\textbf{基于Fit AP架构的P2P即时通信定位系统}}{2015年4月 -- 2015年11月}
\role{小组成员}{个人项目}
\setlength{\parindent}{2em}实现移动端即时通信定位系统
\begin{onehalfspacing}
\begin{itemize}[leftmargin=5em]
  \item 基于Android平台,设计一款P2P即时通信系统,并集成室内定位模块。
  \item 定位模块采用了两种定位方式,包括基于定位锚点的粗粒度定位和基于RSSI信息的较细粒度定位。
  \item 通信模块基于Android平台实现了P2P即时通信功能,为用户提供文字、语音、图片三种基本通信形式。
\end{itemize}
\end{onehalfspacing}

% Reference Test
%\datedsubsection{\textbf{Paper Title\cite{zaharia2012resilient}}}{May. 2015}
%An xxx optimized for xxx\cite{verma2015large}
%\begin{itemize}
%  \item main contribution
%\end{itemize}

\section{\faInfo\ 其他}
% increase linespacing [parsep=0.5ex]
\begin{itemize}[parsep=0.5ex]
  \item \textbf{LinkedIn:}\ https://www.linkedin.com/in/zhlinh
  \item \textbf{GitHub:}\ https://github.com/zhlinh
\end{itemize}

%% Reference
%\newpage
%\bibliographystyle{IEEETran}
%\bibliography{mycite}

\end{document}
